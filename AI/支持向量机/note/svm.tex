\documentclass[UTF8]{ctexart}
\usepackage{amsmath}
\usepackage{lmodern}
\usepackage[none]{hyphenat}
\usepackage{graphicx}
\title{SVM}
\author{Half}
\date{\today}
\begin{document}
\maketitle
\section{间隔和支持向量}
在样本空间中,我们可以将划分超平面转换为
\begin{equation}
    w^Tx+b = 0
\end{equation}
\par 
其中$w =(w_1,w_2..w_n)$,决定了我们的超平面的方向,b表示的是我们的超平面与远点的距离,样本中的任意点x到我们的超平面(w, b)的距离为
$r = \frac{|w^Tx+b|}{||w||}$

假设我们的超平面能够将训练样本正确分类, 即对于$(x_i,y_i)\in D$,若$y_i=+1$,
则有$w^Tx_i+b >0$,若$y_i=-1$则有$w^Tx_i+b<0$,两个异类支持向量到超平面的距离只和为
$r = \frac{2}{||w||}$,我们将他称为margin
\section{对偶问题}
我们希望通过2最大化我们的margin来得到大间隔划分超平面所对应的模型
其中的w和b是我们的模型参数

\section{核函数}

\end{document}