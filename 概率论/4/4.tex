\documentclass[UTF8]{ctexart}
\usepackage{amsmath}
\usepackage{lmodern}
\usepackage[none]{hyphenat}
\usepackage{graphicx}
\title{第四章-随机变量}
\author{程家骏}
\date{\today}
\begin{document}
\maketitle
\section{随机变量}
随机变量指的是在某些条件下取到一些值的可能性,这些定义在样本空间上的实值函数,称为随机变量
\section{离散型随机变量}
如果一个随机变量有多个可能取值,则称这个随机变量为离散型的,对于一个离散型的随机变量X, 我们定义X的概率分布列为
$p(a) = p\{X=a\}$ 同时将保证我们的概率总和为1

\section{期望}
\begin{equation}
    E[x] = \sum_{i=1}^N p(x)
\end{equation}
\section{方差}
同经典概念
\section{伯努利随机变量和二项式随机变量}

考虑一个试验, 其结果分为两类,成功或者是失败,令 x = 1 when success 0 when fail

那么X的分布列为p(0) = P\{X = 0\} = 1 - p,p(1) = p\{X = 1\} = p
假设我们进行n次独立重复性实验, 每次试验成功的概率为p,失败的概率为1-p, 如果x表示的是n次试验中成功的次数,那么称X为参数为(n,p)的二项
随机变量,因此,伯努利随机变量是参数为(1,p)的二项随机变量

参数为(n,p)的二项随机变量的分布列为
\begin{equation}
    p(i) = C_n^i p^i(1-p)^{n-i}
\end{equation}
\section{柏松随机变量}
如果对于一个取值于0, 1, 2...的随机变量对于某一个$\lambda > 0$ 其分布列如下
\begin{equation}
    p(i) = P\{X = i\} = e^{\lambda}\frac{\lambda^i}{i!}\space\space( i = 0,1,2...)
\end{equation}
我们可以得到$\sum_{i=0}^{\infty}p(i) = e^{-\lambda}e^{\lambda} = 1$
\end{document}
